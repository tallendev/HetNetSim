\documentclass[11pt]{article}

\usepackage{fullpage}
\usepackage{amsmath}
\usepackage{setspace}
\doublespacing
\usepackage{fullpage}

\newcommand{\tab}[1]{\hspace{.1\textwidth}\rlap{#1}}

\title{HetNet Simulator: Overview of Basic Mathematical Model}
\author{Matthew Leeds\\
		Tyler Allen\\}
\date{\today}

\begin{document}
\maketitle
This document serves to explain the assumptions and mathematical structure of the model being used in our HetNet Simulator in order to optimize it. It is based on the model used in \cite{amin13} but makes simplifying assumptions.
\section{Assumptions}

\begin{itemize}
\item The system is static-- optimizations are based only on the current state.
\item Devices have the ability to connect to all networks within range.
\item Devices can split traffic across networks in any configuration (they are not restricted to use one radio at a time, or even connect to one of each network type at a time).
\item Devices have no minimum bandwidth requirement; if they can be allocated any nonzero amount, they will be.
\item Signals follow the free space propagation model, meaning a device's achievable data rate is \textit{approximately} the network's maximum data rate times the inverse square of the normalized distance between the device and the AP. 
\item Devices also have no maximum bandwidth rate-- the network's rate is the bottleneck.
\end{itemize}

\section{Mathematics}
\subsection{System Parameters}
\begin{center}
\renewcommand\arraystretch{1.3}
\begin{tabular}{| c | p{12cm} |}
\hline
$A$ & Set of Internet Access Points \\ 
\hline
$U$ & Set of Users \\ 
\hline
$l_a$ & Radius of network $a \in A$ \\
\hline
$d_{ua}$ & Distance between user $u \in U$ and the center of network $a \in A$ \\ 
\hline
$d_{ua,norm}$ & Normalized distance between user $u \in U$ and the center of network $a \in A$, calculated as $\displaystyle\frac{d_{ua}}{l_a}$ \\[8pt] 
\hline
$r_{a,max}$ & Maximum potential rate of network $a \in A$ if it had one user whose $d_{ua} = 0$ \\ 
\hline
$r_{ua,max}$ & Maximum achievable rate by user $u \in U$ on network $a \in A$, calculated as $\displaystyle\frac{r_{a,max}}{(d_{ua,norm})^2}$ whenever $d_{ua,norm} \leq 1$ or 0 otherwise. \\[8pt] 
\hline
$x_{ua}$ & Assignment variable. Detemines the percentage of $r_{ua,max}$ allocated to user $u \in U$ on network $a \in A$ with $0 \leq x_{ua} \leq 1$ \\ 
\hline
$r_{ua}$ & Calculated rate allocated to user $u \in U$ on network $a \in A$, equal to $x_{ua} * r_{ua,max}$ \\ 
\hline
$r_{ua,norm}$ & Normalized rate allocated to user $u \in U$ on network $a \in A$, calculated as $\displaystyle\frac{r_{ua}}{r_{ua,max}}$ \\[8pt] 
\hline
$r_{u}$ & Total rate allocated to user $u \in U$, calculated as $\sum_{a \in A} r_{ua}$ \\ 
\hline
$\alpha$ & User-defined scalar to control the importance of aggregate throughput. $0 \leq \alpha \leq 1$ \\
\hline
$\beta$ & User-defined scalar to control the importance of fairness. $0 \leq \beta \leq 1 - \alpha$ \\
\hline
\end{tabular}
\end{center}
\pagebreak
\subsection{Steps}
\begin{enumerate}
\item Solve the Linear Programming problem:
 
\tab{maximize:}
$$\alpha * \sum_{u \in U} r_u + \beta * z$$
\tab{subject to:}
\begin{equation} \label{eq1}
\sum_{u \in U} r_{ua,norm} \leq 1 \ \ \forall a \in A
\end{equation}
\begin{equation}
0 \leq x_{ua} \leq 1 \ \ \forall u \in U, \forall a \in A
\end{equation}
\begin{equation}
x_{ua} = 0 \text{ whenever } d_{ua,norm} > 1
\end{equation}
\begin{equation}
z \leq r_u \ \ \forall u \in U
\end{equation}

\item Use the solution to calculate $r_{ua} \ \forall u \in U, \ \forall a \in A$ and represent these assignments visually. 
\item Allow the user to adjust $\alpha$ and $\beta$ and observe the effects.
\end{enumerate}

\begin{thebibliography}{9} 
\bibitem{amin13}
"Balancing Spectral Efficiency, Energy Consumption,
and Fairness in Future Heterogeneous Wireless
Systems with Reconfigurable Devices". Rahul Amin, Jim Martin, Juan Deaton, Luiz A. DaSilva, Amr Hussien, and Ahmed Eltawil. IEEE Journal on Selected Areas in Communications, Vol. XX, No. XX, May 2013.
\end{thebibliography}

\end{document}
