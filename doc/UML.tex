\documentclass[11pt]{article}

\usepackage{fullpage}
\usepackage{graphicx}
\usepackage{setspace}
\doublespacing

\begin{document}

\section{Overview}
\subsection{Modern Wireless Connectivity}
Modern Wireless Devices are self-serving. A smart phone can be connected to a
4g network, a 3g network, an LTE network, a Wi-Max network, a WiFi network, 
or any other type of network for which the device contains an appropriate 
radio receiver. The device itself connects to the network chosen by the user. 
This is generally an easy choice for the user; if the 4g network has a 
low signal strength, it is easy to switch to a local wireless network if one 
is available. This seems effective from the perspective of the device owner, 
but having every wireless device owner select their own network of preference 
may not be an optimal setup.

\subsection{A Proprosal}
A better approach may be to have a device controlling the network distribution
in a local area. A device called a Global Resource Controller (GRC) could receive
data about devices and network traffic and make a decision about which network
each individual device should use. This requires what we will call a Heterogenous 
Network (HetNet). HetNets are a composition of all the available wireless 
networks in an area; in our model, a device could make a seamless transition
between these networks at the behest of the GRC. The GRC's job is to ensure 
that thee devices in a HetNet are allocated to networks in an optimal manner.
Optimality could depend on a number of variables; our goal could be to maximize
bandwidth, cost efficiency, or a number of other variables. Our goal is 
to design a tool that can simulate this kind of environment to allow students 
a more visual experience when learning this concept, as well as allowing research
to be done into the feasibility of this kind of setup.

\subsection{Our Tool}
We are in the process of designing a web tool that allows a user to design a 
visual model of a HetNet. We believe that these types of optimality problems
can be represented as Linear Programs and then solved. Our HetNet simulation
should be capable of taking a given HetNet, optimizing it, and returning the 
results in a visual fashion. In the next section, we will discuss the 
design and implementation of this tool.


\section{Tool Design}
\subsection{Overview}
As seen above, our web tool uses a typical web client-server model. The web client
will expose a GUI to the user, allowing them to build a visual model of a 
HetNet. Once the user is satisfied with their model, it will be submitted to the 
server. The server will request a solution to a Linear Program using the information 
from the HetNet model. This Linear Program will then be submitted to our Linear
Program Solver. If all goes well, a solution will be generated and returned to 
the client in some visual fashion. We are in the process of creating an 
error handling system in the event that problems generated are infeasible or 
degenerate. 

\subsection{Implementation}
The client will be created using JavaScript libraries for creating the visual 
elements and a JSON message passing system for communication to the server.
The server will be running a PHP backend which will invoke optimized
C++ code containing the Solver logic. The logic is an implementation of the two-phase
simplex method. We have used the tool \textit{SWIG} to generate  % link to swig?
PHP to C++ bindings. 

\subsection{Solver Design}
The Simplex Solver has been designed in a typical object oriented fashion. 
It specifies a format for Linear Programs which is exposed by the Linear Program
class. Linear Programs can be created and constraints can be added before 
the Linear Program is passed to the Solver. The Solver is an implementation of 
the Singleton design pattern, allowing for a single, lightweight, lazily generated 
object to handle any number of threaded accesses. This is with the intention of 
adding support for "dynamic models"; models that will track devices in the HetNet
over a specific timeframe; this could only be done in a timely manner using 
multiple threads to generate the response. 

\end{document}
